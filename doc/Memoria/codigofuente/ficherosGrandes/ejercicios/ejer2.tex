\section{Ejercicio 2}
\subsection{Enunciado}
\begin{ejer}
    \textbf{2.[itrunc()]} Modifica \texttt{xv6} para manejar el borrado de ficheros con 
    bloques doblemente indirectos.
\end{ejer}

\subsection{Desarrollo}
\subsubsection{xv6/fs.c}
\begin{listing}
@@ -465,8 +466,8 @@ itrunc(struct inode *ip)
itrunc(struct inode *ip)
{
    int i, j;
-   struct buf *bp;
-   uint *a;
+   struct buf *bp, *bp2;
+   uint *a, *b;
@@ -468,12 +486,29 @@ itrunc(struct inode *ip)
        bfree(ip->dev, ip->addrs[NDIRECT]);
        ip->addrs[NDIRECT] = 0;
    }        
+   if (ip->addrs[NDIRECT+1])
+   {
+       bp = bread(ip->dev, ip->addrs[NDIRECT+1]);
+       a = (uint*)bp->data;
+       for(j = 0; j < NINDIRECT; j++)
+       {
+           if(a[j]){
+               bp2 = bread(ip->dev, a[j]);
+               b = (uint*)bp2->data;
+               for(int k = 0; k < NINDIRECT; k++){
+                   if(b[k])
+                       bfree(ip->dev, b[k]);
+               }
+               bfree(ip->dev, a[j]);
+               brelse(bp2);
+           }
+       }
+       brelse(bp);
+       bfree(ip->dev, ip->addrs[NDIRECT+1]);
+       ip->addrs[NDIRECT+1] = 0;
+   }
    ip->size = 0;
    iupdate(ip);
}
\end{listing}

\par A diferencia del primer ejercicio, en este caso tenemos que 
recorrer integramente todas las tablas para ir liberando bloque
a bloque. Para ello, la estrategia a seguir es ir liberando
los bloques de las tablas más profundas y cuando están liberados
todos los bloques, entonces liberar la estrcutura que indexa dichos
bloques.

\subsubsection{Salida del fichero prueba}
\begin{listing}[style=consola]
    $ big
    .........................................................................................
    .......................................................................
    wrote 16523 sectors
    done; ok
    Output code: 0
    $ rm big.file
    Output code: 0
    $ big
    .........................................................................................
    .......................................................................
    wrote 16523 sectors
    done; ok
    Output code: 0
\end{listing}

\par Para probar este ejercicio, hemos ejecutado el programa \texttt{big}, luego
hemos eliminado el archivo generado y ejecutado de nuevo el programa. Si crea
el archivo nuevamente, es que funciona correctamente, ya que si no hubiera liberado
completamente la memoria, no habría suficiente espacio para crearlo. 