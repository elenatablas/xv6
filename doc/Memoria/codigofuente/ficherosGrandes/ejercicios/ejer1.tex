\section{Ejercicio 1}
\subsection{Enunciado}
\begin{ejer}
    \textbf{1.[bmap()]} Modifica \texttt{bmap()} para que implemente un bloque doblemente indirecto. 
    No puedes cambiar el tamaño del \texttt{nodo-i} en disco, por lo que sacrificaremos un bloque directo para 
    implementar el doblemente indirecto.
\end{ejer}

\newpage

\subsection{Desarrollo}
\subsubsection{xv6/param.h}

\begin{listing}
@@ -10,4 +10,4 @@
    #define MAXOPBLOCKS  10  // max # of blocks any FS op writes
    #define LOGSIZE      (MAXOPBLOCKS*3)  // max data blocks in on-disk log
    #define NBUF         (MAXOPBLOCKS*3)  // size of disk block cache
-   #define FSSIZE 1000 // size of file system in blocks
+   #define FSSIZE 20000 
\end{listing}

\par Cambiamos el tamaño máximo del sistema de ficheros.

\subsubsection{xv6/file.h}
\begin{listing}
@@ -22,5 +22,5 @@ struct inode {
    short minor;
    short nlink;
    uint size;
-   uint addrs[NDIRECT+1];
+   uint addrs[NDIRECT+1+1];
  };
\end{listing}

\par Sustituimos un bloque directo del i-nodo para colocar los enlaces
doblemente indirectos en la estructura guardada en memoria.


\subsubsection{xv6/fs.h}
\begin{listing}
@@ -21,3 +21,4 @@ struct superblock {
-   #define NDIRECT 12
+   #define NDIRECT 11
    #define NINDIRECT (BSIZE / sizeof(uint))
+   #define NDINDIRECT (NINDIRECT * NINDIRECT)
-   #define MAXFILE (NDIRECT + NINDIRECT)
+   #define MAXFILE (NDIRECT + NINDIRECT + NDINDIRECT)
@@ -32,5 +33,5 @@ struct dinode {
    short minor;          // Minor device number (T_DEV only)
    short nlink;          // Number of links to inode in file system
    uint size;            // Size of file (bytes)
-   uint addrs[NDIRECT+1];
+   uint addrs[NDIRECT+1+1]; // Data block addresses (Directos + BSI + BDI)
  };
\end{listing}
\begin{center}
\begin{tikzpicture}  
    \matrix[my table] (A) { % First table (it is called A)
       type\\
       mayor\\
       minor\\
       nlink\\
       size\\
       adress 1\\
       .... \\
       \st{adress 12}\\
       indirect \\
       dindirect \\
    };
    \drawLineBelowRow{1}{A}
    \drawLineBelowRow{2}{A}
    \drawLineBelowRow{3}{A}
    \drawLineBelowRow{4}{A}
    \drawLineBelowRow{5}{A}
    \drawLineBelowRow{6}{A}
    \drawLineBelowRow{7}{A}
    \drawLineBelowRow{8}{A}
    \drawLineBelowRow{9}{A}
\end{tikzpicture} 
\par \texttt{struct dinode}
\end{center}
\par Quitamos uno de los bloques directos (quedando en 11 bloques directos)
\par Actualizamos el tamaño maximo posible por un fichero en el sistema
de ficheros.

\par Sustituimos un bloque directo del i-nodo para colocar los enlaces
doblemente indirectos en la estructura guardada en disco.
\newpage

\subsubsection{xv6/fs.c}
\begin{listing}
@@ -375,2 +376,3 @@ bmap(struct inode *ip, uint bn)
-   uint addr, *a;
-   struct buf *bp;
+   uint addr, *a, *b;
+   struct buf *bp, *bp2;
@@ -405,6 +406,31 @@ bmap(struct inode *ip, uint bn)
    if(bn < NINDIRECT){
        brelse(bp);
        return addr;
    }
+    bn -= NINDIRECT;
+    if(bn < NDINDIRECT){
+       if((addr = ip->addrs[NDIRECT+1]) == 0)
+           ip->addrs[NDIRECT+1] = addr = balloc(ip->dev);
+       bp = bread(ip->dev, addr);
+       a = (uint*)bp->data;
+       int global_bn = bn;
+       bn = (global_bn / NINDIRECT);
+       int index_a = bn;
+       if((addr = a[index_a]) == 0){
+           a[index_a] = addr = balloc(ip->dev);
+           log_write(bp);
+       }
+       bp2 = bread(ip->dev, addr);
+       bn = (global_bn % NINDIRECT);
+       int index_b = bn;
+       b = (uint*)bp2->data;
+       if((addr = b[index_b]) == 0){
+           b[index_b] = addr = balloc(ip->dev);
+           log_write(bp2);
+       }
+       brelse(bp2);
+       brelse(bp);
+       return addr;
+   }
    panic("bmap: out of range");
 }
\end{listing}
\begin{center}
	\begin{tikzpicture} 
        \matrix[my table] (D) { 
			.... \\
            NDINDIRECT \\
		};
        \drawLineBelowRow{1}{D}
		\drawLineBelowRow{2}{D}
		\matrix[my table, right=of D] (A) { 
            A \\
			A[index\_a] =address 1\\
			.... \\
			address 128\\
		};
		\drawLineBelowRow{1}{A}
		\drawLineBelowRow{2}{A}
        \drawLineBelowRow{3}{A}
		\matrix[my table, right= of A] (B) { 
            B \\
			B[index\_b] = address 1\\
			.... \\
			address 128\\
		};
		\drawLineBelowRow{1}{B}
		\drawLineBelowRow{2}{B}
        \drawLineBelowRow{3}{B}
        \draw[my arrow] (D-2-1) -- (A-1-1);
		\draw[my arrow] (A-2-1) -- (B-1-1);
	\end{tikzpicture}
	
\end{center}

\par Siguiendo la estructura del método \texttt{bmap()} implementamos la
funcionalidad necesaria para que pueda mapear bloques almacenados
en bloques doblemente indirectos. 

\par Para llegar a la dirección correspondiente tendremos que recorrer las 
dos tablas.

\par Primero tenemos que encontrar el índice \texttt{index\_a} de la primera tabla (A), que
apuntará a la segunda tabla (B). Para ello dividimos el \texttt{bn}
restante entre el número de entradas que tiene la tabla (128).

\par Una vez nos encontramos en la segunda tabla (B), debemos encontrar el índice \texttt{index\_b}
de la segunda tabla, que es el que apuntará a la dirección solicitada.
Para ello usamos el resto de la división \texttt{bn} entre el número de entradas
que tiene la tabla (128). 


\subsection{Ficheros de prueba}

\subsubsection{xv6/Makefile}
\begin{listing}
@@ -194,3 +194,4 @@ UPROGS=\
    _tsbrk4\
    _tsbrk5\
    _tfreem\
+   _big\
\end{listing}

\subsubsection{Salida del fichero prueba}
\begin{listing}[style=consola]
    Booting from Hard Disk...
    xv6...
    cpu0: starting 0
    sb: size 20000 nblocks 19937 ninodes 200 nlog 30 logstart 2 inodestart 32 bmap start 58
    init: starting sh
    $ big
    .........................................................................................
    .......................................................................
    wrote 16523 sectors
    done; ok
    Output code: 0
\end{listing}

\par Si todo va bien, el programa \texttt{big.c} informará que ha podido escribir 
\textbf{16523} sectores. Necesitará bastante tiempo para terminar la ejecución.
