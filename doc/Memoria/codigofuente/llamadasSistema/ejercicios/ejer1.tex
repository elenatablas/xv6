%%%%%%%%%%%%%%%%%%%%%% LLAMADAS AL SISTEMA - EJERCICIO 1 %%%%%%%%%%%%%%%%%%%%%%
\section{Ejercicio 1}
\subsection{Enunciado}
\begin{ejer}
    \textbf{1.[date()]} Añade una nueva llamada al sistema a \texttt{xv6}: \texttt{int date(struct rtcdate *d)}.
\end{ejer}
\subsection{Desarrollo}

\subsubsection{xv6/syscall.h}
\begin{listing}
@@ -20,3 +20,4 @@
    #define SYS_link   19
    #define SYS_mkdir  20
    #define SYS_close  21
+   #define SYS_date   22
\end{listing}
\par Le damos un nuevo número a la llamada.

\subsubsection{xv6/usys.S}
\begin{listing}
@@ -28,4 +28,5 @@
    SYSCALL(getpid)
    SYSCALL(sbrk)
    SYSCALL(sleep)
    SYSCALL(uptime)
+   SYSCALL(date)
\end{listing}
\par Añadimos la llamada \texttt{date}.

\subsubsection{xv6/user.h}
\begin{listing}
@@ -21,6 +21,7 @@ 
// system calls
    extern int dup(int);
    extern int getpid(void);
    extern char* sbrk(int);
    extern int sleep(int);
    extern int uptime(void);
+   extern int date(struct rtcdate*);
\end{listing}
\par Añadimos la llamada \texttt{date()} al fichero de definición de llamadas 
al sistema para los programas de usuario.

\subsubsection{xv6/syscall.c}
\begin{listing}
@@ -101,4 +101,5 @@
    extern int sys_unlink(void);
    extern int sys_wait(void);
    extern int sys_write(void);
    extern int sys_uptime(void);
+   extern int sys_date(void);
@@ -125,5 +126,6 @@ static int (*syscalls[])(void) = {
    [SYS_unlink]  sys_unlink,
    [SYS_link]    sys_link,
    [SYS_mkdir]   sys_mkdir,
    [SYS_close]   sys_close,
+   [SYS_date]    sys_date,
    };
\end{listing}
\par Añadimos la definición de la función \texttt{sys\_date()}.

\subsubsection{xv6/sysproc.c}
\begin{listing}
@@ -92,1 +92,9 @@ int
+   int
+   sys_date(void)
+   {
+       struct rtcdate *d;
+       if(argptr(0, (void **) &d, sizeof(struct rtcdate)) < 0)
+           return -1;
+       cmostime(d);
+       return 0;
+   }
\end{listing}
\par Añadimos la función \texttt{sys\_date()} con su implementación en \texttt{sysproc.c}
donde se implementan las llamadas al sistema que se realizan en \texttt{syscall()}.

\subsection{Fichero de prueba}
\subsubsection{xv6/date.c}
\begin{listing}
@@ +1,19 @@
+   #include "types.h"
+   #include "user.h"
+   #include "date.h"
+
+   int 
+   main(int argc, char *argv[])
+   {
+       struct rtcdate r;
+       if(date(&r))
+       {
+           printf(2, "date failed\n");
+           exit(0);
+       }
+       // Codigo para imprimir la fecha
+       printf(1, "Chiquillo, vaya cabeza que tienes. Hoy estamos a %d/%d/%d\n", r.day, r.month, r.year);
+       exit(0);
+   }
\end{listing}
\par Creamos un programa de usuario que llama a nuestra nueva llamada al sistema \texttt{date()}.

\subsubsection{xv6/Makefile}
\begin{listing}
@@ -185,2 +185,3 @@ UPROGS=\
    _wc\
    _zombie\
+   _date\
\end{listing}
\par Para que nuestro programa esté disponible en el shell de \texttt{xv6}, insertamos 
\texttt{\_date} a la definición \texttt{UPROGS} en el \texttt{Makefile}.

\subsubsection{Salida del fichero prueba}
\begin{listing}[style=consola]
    $ date
    Chiquillo, vaya cabeza que tienes. Hoy estamos a 21/12/2021
    Output code: 0
\end{listing}