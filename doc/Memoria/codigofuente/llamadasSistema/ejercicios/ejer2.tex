%%%%%%%%%%%%%%%%%%%%%% LLAMADAS AL SISTEMA - EJERCICIO 2 %%%%%%%%%%%%%%%%%%%%%%
\section{Ejercicio 2}
\subsection{Enunciado}
\begin{ejer}
    \textbf{2.[dup2()]} Implementa la llamada al sistema \texttt{dup2()} y modifica el shell
para usarla (usa como ejemplo la implementación de la llamada al sistema \texttt{dup()} y consulta cómo
debe de comportarse \texttt{dup2()} según el estándar \texttt{POSIX}).
\end{ejer}
\subsection{Desarrollo}
\subsubsection{xv6/syscall.h}
\begin{listing}
@@ -21,3 +21,4 @@
    #define SYS_mkdir  20
    #define SYS_close  21
    #define SYS_date   22
+   #define SYS_dup2   23
\end{listing}
\par Le damos un nuevo número a la llamada.

\subsubsection{xv6/usys.S}
\begin{listing}
@@ -29,4 +29,5 @@
    SYSCALL(sbrk)
    SYSCALL(sleep)
    SYSCALL(uptime)
    SYSCALL(date)
+   SYSCALL(dup2)
\end{listing}
\par Añadimos la llamada \texttt{dup2}.

\subsubsection{xv6/user.h}
\begin{listing}
@@ -21,7 +21,8 @@ 
// system calls
    extern int dup(int);
    extern int getpid(void);
    extern char* sbrk(int);
    extern int sleep(int);
    extern int uptime(void);
    extern int date(struct rtcdate*);
+   extern int dup2(int, int);
\end{listing}
\par Añadimos la llamada \texttt{dup2()} al fichero de definición de llamadas 
al sistema para los programas de usuario.

\subsubsection{xv6/syscall.c}
\begin{listing}
@@ -103,3 +103,4 @@
    extern int sys_write(void);
    extern int sys_uptime(void);
    extern int sys_date(void);
+   extern int sys_dup2(void);
@@ -127,3 +127,4 @@ static int (*syscalls[])(void) = {
    [SYS_mkdir]   sys_mkdir,
    [SYS_close]   sys_close,
    [SYS_date]    sys_date,
+   [SYS_dup2]    sys_dup2,
    };
\end{listing}
\par Añadimos la definición de la función \texttt{sys\_dup2()}.

\subsubsection{xv6/sysfile.c}
\begin{listing}
@@ -55,1 +55,29 @@ int sys_dup2(void)
+   int
+   sys_dup2(void)
+   {
+       struct file *f;
+       struct file *fbasura;
+       int oldfd, newfd;
+       // comprueba que f este abierto, f = fichero abierto.
+       if(argfd(0, &oldfd, &f) < 0)
+           return -1; // Siempre abierto sino fallo
+       // Obtenemos el segundo argumento, newfd = arg[1];
+       // No hace falta comprobar que esta abierto. Da igual abierto o cerrado
+       if(argint(1, &newfd) < 0)
+           return -1;
+       if(oldfd == newfd)
+           return newfd;
+
+       // Primer fd cerrado y colocar f en ese descriptor
+       if(newfd < 0 || newfd >= NOFILE)
+           return -1;
+       // Fichero anterior porque estaba abierto
+       if((fbasura= myproc()->ofile[newfd]) != 0)
+       {
+           myproc()->ofile[newfd] = 0;
+           fileclose(fbasura);
+       }
+       myproc()->ofile[newfd] = myproc()->ofile[oldfd];
+       // Sumarle uno a la tabla de ficheros del sistema
+       filedup(f);
+       return newfd;
+   }
\end{listing}

\subsection{Fichero de prueba}
\subsubsection{xv6/Makefile}
\begin{listing}
@@ -185,4 +185,5 @@ UPROGS=\
    _wc\
    _zombie\
    _date\
    _clear\
+   _dup2test\
\end{listing}
\subsubsection{Salida del fichero prueba}
\begin{listing}[style=consola]
    $ dup2test
    Este mensaje debe salir por terminal.
    Este mensaje debe salir por terminal (2).
    Este mensaje debe salir por terminal (3).
    Este mensaje debe salir por terminal.
    Output code: 0
\end{listing}