%%%%%%%%%%%%%%%%%%%%%% SISTEMA DE MEMORIA - EJERCICIO 2 %%%%%%%%%%%%%%%%%%%%%%
\section{Ejercicio 2}

\subsection{Enunciado}
\begin{ejer}
    \textbf{2.[sbrk()]} Modifica el código del primer ejercicio para que contemple estas situaciones:
    \begin{itemize}
        \item El caso de un argumento negativo al llamarse a \texttt{sbrk()}.
        \item Manejar el caso de fallos en la página inválida debajo de la pila.
        \item Verificar que \texttt{fork()} y \texttt{exit()/wait()} funciona en el caso de que haya
    direcciones virtuales sin memoria reservada para ellas.
        \item Asegurarse de que funciona el uso por parte del kernel de páginas de usuario que todavía no han sido 
    reservadas (p.e. si un programa pasa una dirección de la zona de usuario todavía no reservada a \texttt{read()}).
    \end{itemize}
\end{ejer}
\subsection{Desarrollo}
\subsubsection{xv6/trap.c}
\begin{listing}
    @@ -79,31 +79,34 @@ trap(struct trapframe *tf)
    case T_PGFLT:
    {
+       if (!(tf->err & 1) && !(rcr2() > myproc()->sz)) { 
\end{listing}

\subsubsection{xv6/sysproc.c}
\begin{listing}
@@ -52,18 +52,32 @@ sys_sbrk(void)
    int
    sys_sbrk(void)
    {
-       int addr;
-       int n; 
+       int n, sz;
+       struct proc *curproc = myproc();
        if(argint(0, &n) < 0)
            return -1;
-       addr = myproc()->sz +=n;
-       myproc()->sz += n;
+       sz = curproc->sz;
+       if(n>=0)
+       {
+           if((sz+n) >= KERNBASE)
+               return -1;
+           curproc->sz = sz+n;
+       }
+       else // argumento negativo
+       {
+           if((curproc->sz = deallocuvm(curproc->pgdir, sz, sz + n)) == 0)
+               return -1;
+       }
+       lcr3(V2P(curproc->pgdir)); // Invalidate TLB.
-       if(growproc(n) < 0)
-           return -1;
-       return addr;
+       return sz; // retornamos el anterior tam
    }
\end{listing}
\par Añadimos el caso de un argumento negativo al llamarse a \texttt{sbrk()}
\par TODO: Manejar el caso de fallos en la página inválida debajo de la pila.
\par TODO: Asegurarse de que funciona el uso por parte del kernel de páginas de 
usuario que todavía no han sido reservadas (p.e. si un programa pasa una dirección 
de la zona de usuario todavía no reservada a \texttt{read()}).

\subsubsection{xv6/vm.c}
\begin{listing}
@@ -323,10 +323,10 @@ copyuvm(pde_t *pgdir, uint sz)
    for(i = 0; i < sz; i += PGSIZE){
        if((pte = walkpgdir(pgdir, (void *) i, 0)) == 0)
-           panic("copyuvm: pte should exist");
+           continue;
        if(!(*pte & PTE_P))
-           panic("copyuvm: page not present");
+           continue;
\end{listing}
\par Para que \texttt{fork()} y \texttt{exit()/wait()} funcionen en el caso de que 
haya direcciones virtuales sin memoria reservada para ellas.

\subsection{Ficheros de prueba}
\subsubsection{Fichero sbrk.c}
\subsubsection{xv6/Makefile}
\begin{listing}
@@ -189,6 +189,11 @@ UPROGS=\
    _dup2test\
    _exitwait\
+   _tsbrk1\
+   _tsbrk2\
+   _tsbrk3\
+   _tsbrk4\
+   _tsbrk5\
\end{listing}

\subsubsection{Salida de los ficheros prueba}
\subsubsection{Salida del fichero prueba}
\begin{listing}[style=consola]
    $ tsbrk1
    pid 3 sh: trap 14 err 6 on cpu 0 eip 0xfa5 addr 0x5004--kill proc
    pid 3 sh: trap 14 err 6 on cpu 0 eip 0x101c addr 0xcfa4--kill proc
    pid 3 tsbrk1: trap 14 err 6 on cpu 0 eip 0x23 addr 0x31f4--kill proc
    pid 3 tsbrk1: trap 14 err 4 on cpu 0 eip 0xb9 addr 0x31f4--kill proc
    Debe imprimir 1: 1.
    Output code: 0
\end{listing}
\begin{listing}[style=consola]
    $ tsbrk2
    pid 4 sh: trap 14 err 6 on cpu 0 eip 0xfa5 addr 0x5004--kill proc
    pid 4 sh: trap 14 err 6 on cpu 0 eip 0x101c addr 0xcfa4--kill proc
    .........................................................................................c
    Output code: 14
\end{listing}
\begin{listing}[style=consola]
    $ tsbrk3
    pid 5 sh: trap 14 err 6 on cpu 0 eip 0xfa5 addr 0x5004--kill proc
    pid 5 sh: trap 14 err 6 on cpu 0 eip 0x101c addr 0xcfa4--kill proc
    pid 5 tsbrk3: trap 14 err 2 on cpu 0 eip 0x8010404f addr 0x5000--kill proc
    Debe imprimir los 50 primeros caracteres de README:
    xv6 is a re-implementation of Dennis Ritchie's and
    Output code: 0
\end{listing}
\begin{listing}[style=consola]
    $ tsbrk4
    pid 6 sh: trap 14 err 6 on cpu 0 eip 0xfa5 addr 0x5004--kill proc
    pid 6 sh: trap 14 err 6 on cpu 0 eip 0x101c addr 0xcfa4--kill proc
    pid 6 tsbrk4: trap 14 err 6 on cpu 0 eip 0x28 addr 0x31f4--kill proc
    pid 6 tsbrk4: trap 14 err 4 on cpu 0 eip 0xbe addr 0x31f4--kill proc
    Debe imprimir 1: 1.
    pid 7 tsbrk4: trap 14 err 6 on cpu 0 eip 0x28 addr 0x31f4--kill proc
    pid 7 tsbrk4: trap 14 err 4 on cpu 0 eip 0xbe addr 0x31f4--kill proc
    Debe imprimir 1: 1.
    pid 7 tsbrk4: trap 14 err 6 on cpu 0 eip 0x100 addr 0x4b3000--kill proc
    pid 7 tsbrk4: trap 14 err 4 on cpu 0 eip 0x11f addr 0x4b3000--kill proc
    Debe imprimir 1: pid 8 tsbrk4: trap 14 err 6 on cpu 0 eip 0x100 addr 0x4b3000--kill proc
    pid 8 tsbrk4: trap 14 err 4 on cpu 0 eip 0x11f addr 0x4b3000--kill proc
    Debe imprimir 1: 1.
    pid 6 tsbrk4: trap 14 err 6 on cpu 0 eip 0x100 addr 0x4b3000--kill proc
    1pid 9 tsbrk4: trap 14 err 6 on cpu 0 eip 0x100 addr 0x4b3000--kill proc
    pid 9 tsbrk4: trap 14 err 4 on cpu 0 eip 0x11f addr 0x4b3000--kill proc
    pid 6 tsbrk4: trap 14 err 4 on cpu 0 eip 0x11f addr 0x4b3000--kill proc
    .
    DeDebe impribe impzombie!
    mir 1:rimir 1: 1.
    1.
    zombie!
    Output code: 0
    zombie!
\end{listing}
\begin{listing}[style=consola]
    $ tsbrk5
    pid 10 sh: trap 14 err 6 on cpu 0 eip 0xfa5 addr 0x5004--kill proc
    pid 10 sh: trap 14 err 6 on cpu 0 eip 0x101c addr 0xcfa4--kill proc
    Este programa primero intenta acceder mas alla de sz.
    Debe fallar ahora:
    pid 10 tsbrk5: trap 14 err 6 on cpu 0 eip 0x23 addr 0x3001--kill proc
    Output code: 14
\end{listing}
\begin{listing}[style=consola]
    $ tsbrk5
    pid 3 sh: trap 14 err 6 on cpu 0 eip 0xfa5 addr 0x5004--kill proc
    pid 3 sh: trap 14 err 6 on cpu 0 eip 0x101c addr 0xcfa4--kill proc
    Este programa primero intenta acceder mas alla de sz.
    Si no fallo antes (mal), ahora tambien debe fallar:
    6687
    Output code: 0
\end{listing}
\begin{listing}[style=consola]
    $ tsbrk5
    pid 3 sh: trap 14 err 6 on cpu 0 eip 0xfa5 addr 0x5004--kill proc
    pid 3 sh: trap 14 err 6 on cpu 0 eip 0x101c addr 0xcfa4--kill proc
    Este programa primero intenta acceder mas alla de sz.
    Si no fallo antes (mal), ahora tambien debe fallar:
    pid 3 tsbrk5: trap 14 err 7 on cpu 0 eip 0x7b addr 0x80000002--kill proc
    Output code: 14
\end{listing}

\subsubsection{Ejecución de tuberías \texttt{ls | wc}}
\begin{listing}[style=consola]
    $:
\end{listing}