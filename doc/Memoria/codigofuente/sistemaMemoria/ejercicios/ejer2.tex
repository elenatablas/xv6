%%%%%%%%%%%%%%%%%%%%%% SISTEMA DE MEMORIA - EJERCICIO 2 %%%%%%%%%%%%%%%%%%%%%%
\section{Ejercicio 2}

\subsection{Enunciado}
\begin{ejer}
    \textbf{2.[sbrk()]} Modifica el código del primer ejercicio para que contemple estas situaciones:
    \begin{itemize}
        \item El caso de un argumento negativo al llamarse a \texttt{sbrk()}.
        \item Manejar el caso de fallos en la página inválida debajo de la pila.
        \item Verificar que \texttt{fork()} y \texttt{exit()/wait()} funciona en el caso de que haya
    direcciones virtuales sin memoria reservada para ellas.
        \item Asegurarse de que funciona el uso por parte del kernel de páginas de usuario que todavía no han sido 
    reservadas (p.e. si un programa pasa una dirección de la zona de usuario todavía no reservada a \texttt{read()}).
    \end{itemize}
\end{ejer}
\subsection{Desarrollo}
\subsubsection{xv6/trap.c}
\begin{listing}
    @@ -79,31 +79,34 @@ trap(struct trapframe *tf)
    case T_PGFLT:
    {
+       if (!(tf->err & 1) && !(rcr2() > myproc()->sz)) { 
\end{listing}

\subsubsection{xv6/sysproc.c}
\begin{listing}
@@ -52,18 +52,32 @@ sys_sbrk(void)
    int
    sys_sbrk(void)
    {
-       int addr;
-       int n; 
+       int n, sz;
+       struct proc *curproc = myproc();
        if(argint(0, &n) < 0)
            return -1;
-       addr = myproc()->sz +=n;
-       myproc()->sz += n;
+       sz = curproc->sz;
+       if(n>=0)
+       {
+           if((sz+n) >= KERNBASE)
+               return -1;
+           curproc->sz = sz+n;
+       }
+       else // argumento negativo
+       {
+           if((curproc->sz = deallocuvm(curproc->pgdir, sz, sz + n)) == 0)
+               return -1;
+       }
+       lcr3(V2P(curproc->pgdir)); // Invalidate TLB.
-       if(growproc(n) < 0)
-           return -1;
-       return addr;
+       return sz; // retornamos el anterior tam
    }
\end{listing}
\par Añadimos el caso de un argumento negativo al llamarse a \texttt{sbrk()}
\par TODO: Manejar el caso de fallos en la página inválida debajo de la pila.
\par TODO: Asegurarse de que funciona el uso por parte del kernel de páginas de 
usuario que todavía no han sido reservadas (p.e. si un programa pasa una dirección 
de la zona de usuario todavía no reservada a \texttt{read()}).

\subsubsection{xv6/vm.c}
\begin{listing}
@@ -323,10 +323,10 @@ copyuvm(pde_t *pgdir, uint sz)
    for(i = 0; i < sz; i += PGSIZE){
        if((pte = walkpgdir(pgdir, (void *) i, 0)) == 0)
-           panic("copyuvm: pte should exist");
+           continue;
        if(!(*pte & PTE_P))
-           panic("copyuvm: page not present");
+           continue;
\end{listing}
\par Para que \texttt{fork()} y \texttt{exit()/wait()} funcionen en el caso de que 
haya direcciones virtuales sin memoria reservada para ellas.

\subsection{Ficheros de prueba}

\subsubsection{xv6/Makefile}
\begin{listing}
@@ -189,6 +189,11 @@ UPROGS=\
    _dup2test\
    _exitwait\
+   _tsbrk1\
+   _tsbrk2\
+   _tsbrk3\
+   _tsbrk4\
+   _tsbrk5\
\end{listing}

\subsubsection{Salida de los ficheros prueba}
\subsubsection{Salida del fichero prueba}
\begin{listing}[style=consola]
    $ tsbrk1
    Debe imprimir 1: 1.
    Output code: 0
\end{listing}
\begin{listing}[style=consola]
    $ tsbrk2
    ..........................................................................................
    .............................................................................pid 3 tsbrk2:
     trap 14 err 7 on cpu 0 eip 0x30a addr 0x2ffc--kill proc
    Output code: 14
\end{listing}
\begin{listing}[style=consola]
    $ tsbrk3
    Debe imprimir los 50 primeros caracteres de README:
    xv6 is a re-implementation of Dennis Ritchie's and
    Output code: 0
\end{listing}
\begin{listing}[style=consola]
    $ tsbrk4
    Debe imprimir 1: 1.
    Debe imprimir 1: 1.
    Debe imprimir 1: 1.
    Debe imprimir 1: Output code: 0
    $Debe imprimir 1: 1.
    1.
    Debe imprimir 1: 1.
    zombie!
    zombie!
    zombie!

    Output code: 0
\end{listing}
\par \texttt{test1();}
\begin{listing}[style=consola]
    $ tsbrk5
    Este programa primero intenta acceder mas alla de sz.
    Debe fallar ahora:
    pid 3 tsbrk5: trap 14 err 6 on cpu 0 eip 0x23 addr 0x3001--kill proc
    Output code: 14
\end{listing}
\par \texttt{test2();}
\begin{listing}[style=consola]
    $ tsbrk5
    Este programa primero intenta acceder mas alla de sz.
    Si no fallo antes (mal), ahora tambien debe fallar:
    6687
    pid 83 tsbrk5: trap 14 err 7 on cpu 0 eip 0x59 addr 0x1a1f--kill proc
    Output code: 14
\end{listing}
\par \texttt{test3();}
\begin{listing}[style=consola]
    $ tsbrk5
    Este programa primero intenta acceder mas alla de sz.
    Si no fallo antes (mal), ahora tambien debe fallar:
    pid 3 tsbrk5: trap 14 err 7 on cpu 0 eip 0x7b addr 0x80000002--kill proc
    Output code: 14
\end{listing}

\subsubsection{Ejecución de tuberías \texttt{ls | wc}}
\begin{listing}[style=consola]
    $ ls | wc
    30 120 736 
    Output code: 0
\end{listing}