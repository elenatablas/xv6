\section{Ejercicio 2}
\subsection{Enunciado}
\begin{ejer}
    \textbf{2.[trap]} Modifica el código del primer ejercicio para que contemple estas situaciones:
    \begin{itemize}
        \item El caso de un argumento negativo al llamarse a \texttt{sbrk()}.
        \item Manejar el caso de fallos en la página inválida debajo de la pila.
        \item Verificar que \texttt{fork()} y \texttt{exit()/wait()} funciona en el caso de que haya
    direcciones virtuales sin memoria reservada para ellas.
        \item Asegurarse de que funciona el uso por parte del kernel de páginas de usuario que todavía no han sido 
    reservadas (p.e. si un programa pasa una dirección de la zona de usuario todavía no reservada a \texttt{read()}).
    \end{itemize}
\end{ejer}
\subsection{Desarrollo}
\begin{listing}
@@
\end{listing}

\subsection{Ficheros de prueba}
\subsubsection{Fichero sbrk.c}
\subsubsection{Salida del fichero prueba}
\begin{listing}[style=consola]
$:
\end{listing}
\subsubsection{Ejecución de tuberías}