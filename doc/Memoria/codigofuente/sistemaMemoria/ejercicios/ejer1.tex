%%%%%%%%%%%%%%%%%%%%%% SISTEMA DE MEMORIA - EJERCICIO 1 %%%%%%%%%%%%%%%%%%%%%%
\section{Ejercicio 1}

\subsection{Enunciado}
\begin{ejer}
    \textbf{1.[sbrk()]} Implementa la característica de reserva diferida en \texttt{xv6}.
\end{ejer}
\subsection{Desarrollo}
\subsubsection{xv6/sysproc.c}
\begin{listing}
@@ -57,6 +57,6 @@ sys_sbrk(void)
    if(argint(0, &n) < 0)
        return -1;
-   if(growproc(n) < 0)
-      return -1;
+   addr = myproc()->sz +=n;
+   myproc()->sz += n;
    return addr;
}
\end{listing}
\par Elimina la reserva de páginas de la llamada al sistema \texttt{sbrk()}, 
implementada a través de la función \texttt{sys\_sbrk()} en \texttt{sysproc.c}.
\par No llama a \texttt{growproc()} en caso de que el proceso crezca e incrementa el tamaño del proceso \texttt{myproc()->sz}.
Devuelve el tamaño antiguo, pero no reserva memoria.
\newpage
\subsubsection{xv6/trap.c}
\begin{listing}
@@ -77,9 +77,30 @@ trap(struct trapframe *tf)
    case T_IRQ0 + IRQ_SPURIOUS:
        cprintf("cpu%d: spurious interrupt at %x:%x\n",
                cpuid(), tf->cs, tf->eip);
        lapiceoi();
        break;
+   case T_PGFLT:
+   {
+       pde_t *pgdir = myproc()->pgdir;
+       int oldsz = rcr2();
+       int newsz = oldsz+PGSIZE;
+       uint a = PGROUNDDOWN(oldsz); 
+       void *mem = kalloc();
+       if(mem == 0){
+           cprintf("allocuvm out of memory\n");
+           deallocuvm(pgdir, newsz, oldsz);
+           break;
+       }
+       memset(mem, 0, PGSIZE);
+       if(mappages(pgdir, (char*)a, PGSIZE, V2P(mem), PTE_W|PTE_U) < 0){
+           cprintf("allocuvm out of memory (2)\n");
+           deallocuvm(pgdir, newsz, oldsz);
+           kfree(mem);
+           break;
+       }
+       break;
+   }
    //PAGEBREAK: 13
    default:
        if(myproc() == 0 || (tf->cs&3) ==0){
            // In kernel, it must be our mistake.
            cprintf("unexpected trap %d from cpu %d eip %x (cr2=0x%x)\n", tf->trapno, cpuid(), tf->eip, rcr2());
\end{listing}
\par Modificamos el código en \texttt{trap.c} para responder a un fallo de página en 
el espacio de usuario mapeando una nueva página física en la dirección que generó el fallo,
regresando después al espacio de usuario para que continue.
\par En \texttt{rcr2()} se encuentra la dirección que ha provocado el fallo de página y
tenemos que mapear.
\par La implementación hace uso de \texttt{PGROUNDDOWN(oldsz)} para alinear las direcciones
de memoria al inicio de la página que corresponda.

\newpage

\subsubsection{xv6/vm.c}
\begin{listing}
@@ -60,4 +60,4 @@ int
-   static int
+   int
    mappages(pde_t *pgdir, void *va, uint size, uint pa, int perm)
    {
        char *a, *last;
\end{listing}
\par Borramos la declaración de función estática en \texttt{vm.c} para poder utilizarla en el 
fichero \texttt{trap.c}.

\subsection{Pruebas}
\subsubsection{Comando \texttt{echo}}
\begin{listing}[style=consola]
    $ echo De ASO no paso, necesito un repaso.
    De ASO no paso, necesito un repaso.
    Output code: 0
\end{listing}
\subsubsection{Comando \texttt{ls}}
\begin{listing}[style=consola]
    $ ls
    .              1 1 512
    ..             1 1 512
    README         2 2 2327
    cat            2 3 3072
    echo           2 4 2772
    forktest       2 5 1944
    grep           2 6 3624
    init           2 7 3052
    kill           2 8 2800
    ln             2 9 2812
    ls             2 10 3604
    mkdir          2 11 2856
    rm             2 12 2848
    sh             2 13 6396
    stressfs       2 14 3008
    usertests      2 15 25596
    wc             2 16 3176
    zombie         2 17 2716
    date           2 18 2836
    clear          2 19 2712
    dup2test       2 20 4228
    exitwait       2 21 3224
    Output code: 0
\end{listing}