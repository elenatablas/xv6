\section{Ejercicio 1}
\subsection{Enunciado}
\begin{ejer}
    \textbf{1.[trap]} Implementa la característica de reserva diferida en \texttt{xv6}.
\end{ejer}
\subsection{Desarrollo}
\begin{listing}
@@
\end{listing}
\subsection{Ficheros de prueba}

\subsubsection{Salida del fichero prueba}
\begin{listing}[style=consola]
    $ tsbrk1
    pid 3 sh: trap 14 err 6 on cpu 0 eip 0xfa5 addr 0x5004--kill proc
    pid 3 sh: trap 14 err 6 on cpu 0 eip 0x101c addr 0xcfa4--kill proc
    pid 3 tsbrk1: trap 14 err 6 on cpu 0 eip 0x23 addr 0x31f4--kill proc
    pid 3 tsbrk1: trap 14 err 4 on cpu 0 eip 0xb9 addr 0x31f4--kill proc
    Debe imprimir 1: 1.
    Output code: 0
\end{listing}
\begin{listing}[style=consola]
    $ tsbrk2
    pid 4 sh: trap 14 err 6 on cpu 0 eip 0xfa5 addr 0x5004--kill proc
    pid 4 sh: trap 14 err 6 on cpu 0 eip 0x101c addr 0xcfa4--kill proc
    .........................................................................................c
    Output code: 14
\end{listing}
\begin{listing}[style=consola]
    $ tsbrk3
    pid 5 sh: trap 14 err 6 on cpu 0 eip 0xfa5 addr 0x5004--kill proc
    pid 5 sh: trap 14 err 6 on cpu 0 eip 0x101c addr 0xcfa4--kill proc
    pid 5 tsbrk3: trap 14 err 2 on cpu 0 eip 0x8010404f addr 0x5000--kill proc
    Debe imprimir los 50 primeros caracteres de README:
    xv6 is a re-implementation of Dennis Ritchie's and
    Output code: 0
\end{listing}
\begin{listing}[style=consola]
    $ tsbrk4
    pid 6 sh: trap 14 err 6 on cpu 0 eip 0xfa5 addr 0x5004--kill proc
    pid 6 sh: trap 14 err 6 on cpu 0 eip 0x101c addr 0xcfa4--kill proc
    pid 6 tsbrk4: trap 14 err 6 on cpu 0 eip 0x28 addr 0x31f4--kill proc
    pid 6 tsbrk4: trap 14 err 4 on cpu 0 eip 0xbe addr 0x31f4--kill proc
    Debe imprimir 1: 1.
    pid 7 tsbrk4: trap 14 err 6 on cpu 0 eip 0x28 addr 0x31f4--kill proc
    pid 7 tsbrk4: trap 14 err 4 on cpu 0 eip 0xbe addr 0x31f4--kill proc
    Debe imprimir 1: 1.
    pid 7 tsbrk4: trap 14 err 6 on cpu 0 eip 0x100 addr 0x4b3000--kill proc
    pid 7 tsbrk4: trap 14 err 4 on cpu 0 eip 0x11f addr 0x4b3000--kill proc
    Debe imprimir 1: pid 8 tsbrk4: trap 14 err 6 on cpu 0 eip 0x100 addr 0x4b3000--kill proc
    pid 8 tsbrk4: trap 14 err 4 on cpu 0 eip 0x11f addr 0x4b3000--kill proc
    Debe imprimir 1: 1.
    pid 6 tsbrk4: trap 14 err 6 on cpu 0 eip 0x100 addr 0x4b3000--kill proc
    1pid 9 tsbrk4: trap 14 err 6 on cpu 0 eip 0x100 addr 0x4b3000--kill proc
    pid 9 tsbrk4: trap 14 err 4 on cpu 0 eip 0x11f addr 0x4b3000--kill proc
    pid 6 tsbrk4: trap 14 err 4 on cpu 0 eip 0x11f addr 0x4b3000--kill proc
    .
    DeDebe impribe impzombie!
    mir 1:rimir 1: 1.
    1.
    zombie!
    Output code: 0
    zombie!
\end{listing}
\begin{listing}[style=consola]
    $ tsbrk5
    pid 10 sh: trap 14 err 6 on cpu 0 eip 0xfa5 addr 0x5004--kill proc
    pid 10 sh: trap 14 err 6 on cpu 0 eip 0x101c addr 0xcfa4--kill proc
    Este programa primero intenta acceder mas alla de sz.
    Debe fallar ahora:
    pid 10 tsbrk5: trap 14 err 6 on cpu 0 eip 0x23 addr 0x3001--kill proc
    Output code: 14
\end{listing}
\begin{listing}[style=consola]
    $ tsbrk5
    pid 3 sh: trap 14 err 6 on cpu 0 eip 0xfa5 addr 0x5004--kill proc
    pid 3 sh: trap 14 err 6 on cpu 0 eip 0x101c addr 0xcfa4--kill proc
    Este programa primero intenta acceder mas alla de sz.
    Si no fallo antes (mal), ahora tambien debe fallar:
    6687
    Output code: 0
\end{listing}
\begin{listing}[style=consola]
    $ tsbrk5
    pid 3 sh: trap 14 err 6 on cpu 0 eip 0xfa5 addr 0x5004--kill proc
    pid 3 sh: trap 14 err 6 on cpu 0 eip 0x101c addr 0xcfa4--kill proc
    Este programa primero intenta acceder mas alla de sz.
    Si no fallo antes (mal), ahora tambien debe fallar:
    pid 3 tsbrk5: trap 14 err 7 on cpu 0 eip 0x7b addr 0x80000002--kill proc
    Output code: 14
\end{listing}

