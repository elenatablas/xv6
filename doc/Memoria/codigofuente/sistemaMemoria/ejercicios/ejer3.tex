%%%%%%%%%%%%%%%%%%%%%% SISTEMA DE MEMORIA - EJERCICIO 3 %%%%%%%%%%%%%%%%%%%%%%
\section{Ejercicio 3}

\subsection{Enunciado}
\begin{ejer}
    \textbf{3.[exec()]} Modifica el código de la función \texttt{exec()} para que asigne 
    a cada proceso un número de páginas de pila igual al número de páginas de \texttt{código+datos} 
    de ese proceso. Compruebe que las anteriores modificaciones siguen funcionando.
\end{ejer}

\subsection{Desarrollo}

\subsubsection{xv6/exec.c}
\begin{listing}
@@ -68,5 +68,7 @@ exec(char *path, char **argv)
    sz = PGROUNDUP(sz);
+   uint sz_cod_datos = sz + PGSIZE;
-   if((sz = allocuvm(pgdir, sz, sz + 2*PGSIZE)) == 0)    
+   if((sz = allocuvm(pgdir, sz, sz + sz_cod_datos)) == 0)
        goto bad;
-   clearpteu(pgdir, (char*)(sz - 2*PGSIZE));
+   clearpteu(pgdir, (char*)(sz - sz_cod_datos));
    sp = sz;
+   curproc->heap = sp;
\end{listing}

\par Para hacer este ejercicio, lo único que tenemos que hacer es modificar
dentro del \texttt{exec()} la forma en la que se va creando el mapa de memoria.
\par En este caso, ajustamos el código para que cree las páginas de pila correspondientes
a codigo más datos, que en este caso seria la varible \texttt{sz} que es la cantidad de datos que ya llevamos.
\par A \texttt{sz} le añadimos tamaño de página \texttt{PGSIZE} para añadir la página de guarda.
\par También utilizamos \texttt{clearpteu()} para proteger la página de guarda y que provoque
una excepción al tratar de acceder a ella.

\subsubsection{xv6/proc.h}
\begin{listing}
@@ -50,4 +50,5 @@ struct proc {
        struct inode *cwd;           // Current directory
        char name[16];               // Process name (debugging)
        int exit_status;             // Estatus de salida
+       uint heap;   
    };
\end{listing}
\par Para poder guardar el comienzo de la memoria del \texttt{heap} del proceso tuvimos que añadir otro campo a la estructura \texttt{struct proc}.

\subsubsection{xv6/sysproc.c}
\begin{listing}
@@ -70,6 +70,10 @@ sys_sbrk(void)
    else // argumento negativo
    {
        if((curproc->sz = deallocuvm(curproc->pgdir, sz, sz + n)) == 0)
            return -1;
+       if(sz+n < myproc()->heap)
+       {
+           return -1;
+       }
    }
    lcr3(V2P(curproc->pgdir));  // Invalidate TLB.
\end{listing}

\par Verificamos que no se libere memoria por debajo del \texttt{heap}.