%%%%%%%%%%%%%%%%%%%%%% SISTEMA DE MEMORIA - EJERCICIO 3 %%%%%%%%%%%%%%%%%%%%%%
\section{Ejercicio 3}

\subsection{Enunciado}
\begin{ejer}
    \textbf{3.[exec()]} Modifica el código de la función \texttt{exec()} para que asigne 
    a cada proceso un número de páginas de pila igual al número de páginas de \texttt{código+datos} 
    de ese proceso. Compruebe que las anteriores modificaciones siguen funcionando.
\end{ejer}

\subsection{Desarrollo}

\subsubsection{xv6/exec.c}
\begin{listing}
@@ -68,6 +68,8 @@ exec(char *path, char **argv)
    sz = PGROUNDUP(sz);
-   if((sz = allocuvm(pgdir, sz, sz + 2*PGSIZE)) == 0)    
+   if((sz = allocuvm(pgdir, sz, sz + (sz+PGSIZE))) == 0)
        goto bad;
-   clearpteu(pgdir, (char*)(sz - 2*PGSIZE));
+   clearpteu(pgdir, (char*)(sz - sz*PGSIZE));
    sp = sz;
+   curproc->heap = sp;
\end{listing}

\subsubsection{xv6/proc.h}
\begin{listing}
@@ -50,6 +50,7 @@ struct proc {
        struct inode *cwd;           // Current directory
        char name[16];               // Process name (debugging)
        int exit_status;             // Estatus de salida
+       uint heap;   
    };
\end{listing}

\subsubsection{xv6/sysproc.c}
\begin{listing}
@@ -70,22 +70,25 @@ sys_sbrk(void)
    else // argumento negativo
    {
        if((curproc->sz = deallocuvm(curproc->pgdir, sz, sz + n)) == 0)
            return -1;
+       if(sz+n < myproc()->heap)
+       {
+           return -1;
+       }
    }
    lcr3(V2P(curproc->pgdir));  // Invalidate TLB.
\end{listing}
