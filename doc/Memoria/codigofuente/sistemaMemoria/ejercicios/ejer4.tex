%%%%%%%%%%%%%%%%%%%%%% SISTEMA DE MEMORIA - EJERCICIO 4 %%%%%%%%%%%%%%%%%%%%%%
\section{Ejercicio 4}

\subsection{Enunciado}
\begin{ejer}
    \textbf{4.[freemem()]} Añade una llamada al sistema \texttt{freemem} con la siguiente 
    signatura: \texttt{int freemem(int type)}.
\end{ejer}

\subsection{Desarrollo}

\subsubsection{xv6/syscall.h}
\begin{listing}
@@ -23,2 +23,3 @@
    #define SYS_date   22
    #define SYS_dup2   23
+   #define SYS_freemem 24
\end{listing}
\par Le damos un nuevo número a la llamada.

\subsubsection{xv6/usys.S}
\begin{listing}
@@ -31,3 +31,4 @@
    SYSCALL(uptime)
    SYSCALL(date)
    SYSCALL(dup2)
+   SYSCALL(freemem)
\end{listing}
\par Añadimos la llamada \texttt{freemem}.

\subsubsection{xv6/user.h}
\begin{listing}
@@ -3,3 +3,5 @@
    #define WIFSIGNALED(status) (((status) & 0x7f) != 0)
    #define WEXITTRAP(status)   (((status) & 0x7f)-1)

+   #define F_PAGES 0
+   #define F_BYTES 1
@@ -44,4 +47,5 @@ 
// system calls
    extern int uptime(void);
    extern int date(struct rtcdate*);
    extern int dup2(int, int);
+   extern int freemem(int);
\end{listing}

\par Definimos las constantes \texttt{F\_PAGES} y \texttt{F\_BYTES} que
son usadas para solicitar a \texttt{freemem()} si deseamos el número de páginas de memoria física
disponibles o el número de bytes de memoria física disponibles.

\par Añadimos la llamada \texttt{freemem()} al fichero de definición de llamadas 
al sistema para los programas de usuario.


\subsubsection{xv6/syscall.c}
\begin{listing}
@@ -105,3 +105,4 @@
    extern int sys_uptime(void);
    extern int sys_date(void);
    extern int sys_dup2(void);
+   extern int sys_freemem(void);
@@ -130,4 +131,5 @@ static int (*syscalls[])(void) = {
    [SYS_close]   sys_close,
    [SYS_date]    sys_date,
    [SYS_dup2]    sys_dup2,
+   [SYS_freemem] sys_freemem,
    };
\end{listing}
\par Añadimos la definición de la función \texttt{sys\_freemem()}.

\subsubsection{xv6/sysproc.c}
\begin{listing}
@@ -85,1 +85,16 @@ int
+   int 
+   sys_freemem (void)
+   {
+       int type;
+       if(argint(0, &type) < 0)
+       {
+           return -1;
+       }
+       if(type == F_PAGES)
+       {
+           return getNumFreePages();
+       }else if(type == F_BYTES){
+           return (getNumFreePages()*PGSIZE);
+       }
+       return -1;
+   }
\end{listing}
\par Añadimos la función \texttt{sys\_freemem()} con su implementación en \texttt{sysproc.c}
donde se implementan las llamadas al sistema que se realizan en \texttt{syscall()}.

\par La llamada al sistema usa el método \texttt{getNumFreePages()} que
devuelve el número de páginas físicas disponibles.

\par En caso de que se desee obtener el número de bytes, multiplicamos el número de páginas físicas disponibles
por el tamaño de página en bytes.

\subsubsection{xv6/defs.h}
\begin{listing}
@@ -68,4 +68,5 @@ 
// kalloc.c
    void            kfree(char*);
    void            kinit1(void*, void*);
    void            kinit2(void*, void*);
+   int             getNumFreePages(void);
@@ -187,2 +187,3 @@
// vm.c
    void clearpteu(pde_t *pgdir, char *uva);
+   int mappages(pde_t *pgdir, void *va, uint size, uint pa, int perm);
@@ -194,1 +194,3 @@ 
    #define NULL 0
+   #define F_PAGES 0
+   #define F_BYTES 1
\end{listing}

\par Aquí, al igual que en \texttt{xv6/user.h} definimos las constantes
\texttt{F\_PAGES} y \texttt{F\_BYTES} para poder utilizarlas en el código
del sistema operativo.

\subsubsection{xv6/kalloc.c}
\begin{listing}
@@ -21,4 +21,5 @@ struct {
    struct spinlock lock;
    int use_lock;
    struct run *freelist;
+   int num_free;
} kmem;
@@ -32,7 +32,8 @@ kinit1(void *vstart, void *vend)
void
kinit1(void *vstart, void *vend)
{
    initlock(&kmem.lock, "kmem");
    kmem.use_lock = 0;
+   kmem.num_free = 0; //Inicializamos kmem.num_free al principio
    freerange(vstart, vend);
}
@@ -72,6 +73,7 @@ kfree(char *v)
    r = (struct run*)v;
    r->next = kmem.freelist;
    kmem.freelist = r;
+   kmem.num_free++;
    if(kmem.use_lock)
        release(&kmem.lock);
}
@@ -88,6 +90,15 @@ kalloc(void)
    r = kmem.freelist;
    if(r)
+   {
        kmem.freelist = r->next;
+       kmem.num_free--;
+   }
    if(kmem.use_lock)
    release(&kmem.lock);
    return (char*)r;
    }
+   int
+   getNumFreePages()
+   {
+       return kmem.num_free;
+   }
\end{listing}

\par Creamos en la estructura \texttt{kmem} el atributo \texttt{num\_free}
que es un contador de número de páginas físicas disponibles.

\par Es inicializado en el método \texttt{kinit1()} y actualiza su valor
en el método \texttt{kalloc()} y \texttt{kfree()}.

\par Por último, usamos el método \texttt{getNumFreePages()} para poder 
obtener el valor de \texttt{num\_free}.

\subsection{Ficheros de prueba}
\begin{listing}
@@ -194,2 +194,3 @@ UPROGS=\
    _tsbrk4\
    _tsbrk5\
+   _tfreem\
\end{listing}
\par Para que nuestro programa esté disponible en el shell de \texttt{xv6}, insertamos 
\texttt{\_tfreem} a la definición \texttt{UPROGS} en el \texttt{Makefile}.

\subsubsection{Salida del fichero prueba}
\begin{listing}[style=consola]
    $ tfreem
    Paginas libres: 56789
    Bytes libres: 232607744
    Output code: 0
\end{listing}