%%%%%%%%%%%%%%%%%%%%%% SISTEMA DE MEMORIA - EJERCICIO 4 %%%%%%%%%%%%%%%%%%%%%%
\section{Ejercicio 4}

\subsection{Enunciado}
\begin{ejer}
    \textbf{4.[freemem()]} Añade una llamada al sistema \texttt{freemem} con la siguiente 
    signatura: \texttt{int freemem(int type)}.
\end{ejer}

\subsection{Desarrollo}

\subsubsection{xv6/syscall.h}
\begin{listing}
@@ -23,2 +23,3 @@
    #define SYS_date   22
    #define SYS_dup2   23
+   #define SYS_freemem 24
\end{listing}

\subsubsection{xv6/usys.S}
\begin{listing}
    SYSCALL(uptime)
    SYSCALL(date)
    SYSCALL(dup2)
+   SYSCALL(freemem)
\end{listing}

\subsubsection{xv6/user.h}
\begin{listing}
@@ -3,6 +3,9 @@
    #define WIFSIGNALED(status) (((status) & 0x7f) != 0)
    #define WEXITTRAP(status)   (((status) & 0x7f)-1)

+   #define F_PAGES 0
+   #define F_BYTES 1
@@ -44,6 +47,7 @@ 
// system calls
    extern int uptime(void);
    extern int date(struct rtcdate*);
    extern int dup2(int, int);
+   extern int freemem(int);
\end{listing}

\subsubsection{xv6/syscall.c}
\begin{listing}
@@ -105,6 +105,7 @@
    extern int sys_uptime(void);
    extern int sys_date(void);
    extern int sys_dup2(void);
+   extern int sys_freemem(void);
@@ -130,6 +131,7 @@ static int (*syscalls[])(void) = {
    [SYS_close]   sys_close,
    [SYS_date]    sys_date,
    [SYS_dup2]    sys_dup2,
+   [SYS_freemem] sys_freemem,
    };
\end{listing}

\subsubsection{xv6/sysproc.c}
\begin{listing}
@@ -85,1 +85,15 @@ int
+   int 
+   sys_freemem (void)
+   {
+       int type;
+       if(argint(0, &type) < 0)
+       {
+           return -1;
+       }
+       if(type == 0)
+       {
+           return getNumFreePages();
+       }
+       return (getNumFreePages()*PGSIZE);
+   }
\end{listing}

\subsubsection{xv6/defs.h}
\begin{listing}
@@ -68,6 +68,7 @@ 
// kalloc.c
    void            kfree(char*);
    void            kinit1(void*, void*);
    void            kinit2(void*, void*);
+   int             getNumFreePages(void);
@@ -187,6 +187,7 @@
// vm.c
    void clearpteu(pde_t *pgdir, char *uva);
+   int mappages(pde_t *pgdir, void *va, uint size, uint pa, int perm);
\end{listing}

\subsubsection{xv6/kalloc.c}
\begin{listing}
@@ -21,6 +21,7 @@ struct {
    struct spinlock lock;
    int use_lock;
    struct run *freelist;
+   int num_free;
} kmem;
@@ -72,6 +73,7 @@ kfree(char *v)
    r = (struct run*)v;
    r->next = kmem.freelist;
    kmem.freelist = r;
+   kmem.num_free++;
    if(kmem.use_lock)
        release(&kmem.lock);
}
@@ -88,9 +90,17 @@ kalloc(void)
    r = kmem.freelist;
    if(r)
+   {
        kmem.freelist = r->next;
+       kmem.num_free--;
+   }
    if(kmem.use_lock)
    release(&kmem.lock);
    return (char*)r;
    }
+   int
+   getNumFreePages()
+   {
+       return kmem.num_free;
+   }
\end{listing}


\subsection{Ficheros de prueba}
\begin{listing}
@@ -194,6 +194,7 @@ UPROGS=\
    _tsbrk4\
    _tsbrk5\
+   _tfreem\
\end{listing}

\subsubsection{Salida del fichero prueba}
\begin{listing}[style=consola]
    $ tfreem
    pid 68 sh: trap 14 err 6 on cpu 0 eip 0xfa5 addr 0x5004--kill proc
    pid 68 sh: trap 14 err 6 on cpu 0 eip 0x101c addr 0xcfa4--kill proc
    Paginas libres: 56789
    Bytes libres: 232607744
    Output code: 0
\end{listing}